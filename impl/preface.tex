\secrel{Предисловие}

\subsecly{Система Little Smalltalk: немного истории}

Весной 1984 года я преподавал курс по языкам программирования в университете Аризоны. 
При подготовке лекций для этого курса я заинтересовался концепцией объектно-ориентированного 
программирования и, в частности, тем, как объектно-ориентированная парадигма изменила 
подход программистов к решению проблем. В течение этого семестра и следующего лета я 
собрал как можно больше материалов об объектно-ориентированном программировании, 
особенно о системе программирования \st-80, разработанной в исследовательском 
центре Xerox Palo Alto (Xerox PARC). Тем не менее, я продолжал расстраиваться из-за 
своей неспособности получить практический опыт написания и использования программ на \st.

В то время единственной системой \st, о которой я знал, была оригинальная система, 
работающая на Dorado, дорогой машине, недоступной (в то время) за пределами Xerox PARC. 
Доступной мне возможностью был VAX-780 с Unix2 использовавший обычные ASCII терминалы. 
Таким образом, оказалось, что мои шансы на запуск системы Xerox \st-80 в 
ближайшей перспективе были весьма невелики; поэтому, несколько студентов и я 
решили летом 1984 года создать нашу собственную систему \st.

Осенью 1984 года мы с дюжиной студентов создали систему Little Smalltalk в рамках 
семинара для выпускников по реализации языка программирования. С самого начала наши 
цели были гораздо менее амбициозными, чем у первоначальных разработчиков системы 
\st-80. Несмотря на то, что мы оценили важность инновационных концепций в 
средах программирования и графике, впервые разработанных группой Xerox, мы до боли 
осознавали наши собственные ограничения, как в рабочей силе, так и в оборудовании. 
Нашими целями в порядке важности были:
\begin{itemize}[nosep]
    \item Новая система должна поддерживать язык, максимально приближенный к 
    опубликованному описанию \st-80 \cite{blue}.
    \item Система должна работать под Unix, используя только обычные текстовые терминалы.
    \item Система должна быть написана на \ci\ и быть максимально переносимой.
    \item Система должна быть маленькой. В частности, она должна работать на 
    16-битных машинах с раздельной памятью команд и данных, 
    но предпочтительно даже на машинах без этой функции.
\end{itemize}

\noindent
Оглядываясь назад, мы, кажется, достигли наших целей довольно хорошо. Язык, понимаемый 
системой Little Smalltalk, достаточно близок к языку системы программирования 
\st-80, так что пользователи, похоже, испытывают только небольшие трудности 
(по крайней мере, с языком) при переходе от одной системы к другой. Система 
оказалась чрезвычайно переносимой: она была перенесена на дюжину разновидностей 
Unix, работающих на разных машинах. Более 200 сайтов теперь используют 
систему Little Smalltalk.

\subsecly{О системе Little Smalltalk}

Эта книга состоит из двух частей. Первый раздел описывает язык системы Little 
Smalltalk. Хотя большинство читателей, возможно, до знакомства с Smalltalk 
имели некоторый опыт использования хотя бы одного другого языка программирования, 
в тексте не делается никаких предположений относительно подготовки читателя. 
Большинство студентов старших курсов или аспирантов должны быть в состоянии 
понять материал в первом разделе. Эта часть текста может использоваться отдельно.

Вторая часть книги описывает фактическую реализацию системы Little Smalltalk. 
Этот раздел требует от читателя более глубоких знаний в области информатики. 
Поскольку Little Smalltalk написан на C, требуется хотя бы элементарное знание 
этого языка. Также желательная хорошая подготовка по структурам данных. 
Читателю будет желательно, хотя и не обязательно, иметь некоторое 
представление о построении компиляторов для обычного языка, такого как Pascal.

\subsecly{Благодарности}

\ldots

\subsecly{Получение оригинальной системы Little Smalltalk}

\url{https://github.com/crcx/littlesmalltalk} Архив Little Smalltalk 
(с обновлениями для работы на современных платформах). 
Он также собирает форки и документацию по этой исторической системе.

\bigskip

The Little Smalltalk system can be obtained directly from the author. The
system is distributed on 9-track tapes in tar format (the standard unix
distribution format). The distribution tape includes all sources and on-line
documentation for the system. For further information on the distribution,
including cost, write to the following address:

\bigskip\noindent
Smalltalk Distribution\\
Department of Computer Science\\
Oregon State University\\
Corvallis, Oregon\\
97331\\
USA
