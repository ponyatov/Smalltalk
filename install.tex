\clearpage
\secly{Установка}\label{install}

\subsecly{\linux}

\begin{verbatim}
# Debian GNU/Linux v.9 stretch x86_64
$ cd ~
~$ git clone -o gh https://github.com/ponyatov/Smalltalk.git
~$ cd Smalltalk
~/Smalltalk$ make install
~/Smalltalk$ ./rc.rc
\end{verbatim}
\lst{rc.rc}{rc.rc}

\subsecly{\win}

Для разработки под \win\ кроме версии \st\ Pharo нужно установить
\href{http://www.mingw.org/}{MinGW/MSYS}\ --- реализация пакета компиляторов и утилит GNU toolchain
для \win.

\begin{lstlisting}
PATH = C:\MinGW\bin;C:\MinGW\msys\1.0\bin;%PATH%
\end{lstlisting}

\begin{description}[nosep]
    \item{\file{mingw32-gcc}} компилятор \ci
    \item{\file{mingw32-gcc-g++}} компилятор \cpp
    % \item{\file{msys-base}}
    \item{\file{msys-make}} утилита автоматической сборки проектов \href{http://www.opennet.ru/docs/RUS/gnumake/}{GNU Make}
    \item{\file{msys-wget}} утилита загрузки по ftp/http
    \item{\file{msys-unzip}}
\end{description}

\medskip\noindent
Руководство включает набор файлов, который распространяется через \href{https://github.com/ponyatov/Smalltalk}{GitHub},
для загрузки и обновления требуется установленный клиент \url{https://git-scm.com}

\clearpage
\subsecly{\ph: системные файлы (после прочтения съесть!)}\label{image}

Базовая установка \ph\ состоит из четырех компонентов:

\begin{description}
    \item{виртуальная машина (VM)}\\ только этот компонент отличается для разных \term{хост-ОС}, это исполняющая система аналогичная JVM.
    Оригинальный \st\ интерпретировал \term{байт-код}, в который компилируются куски кода. \ph\ комплектуется
    \href{http://www.mirandabanda.org/cogblog/about-cog/}{CogVM}, быстрой виртуальной машиной,
    использующей оптимизации и компиляцию в машинный код (JIT)

    \begin{tabular}{l l}
    \verb|pharo.exe| & для \win\\
    \verb|pharo|     & \linux\\
    \verb|Pharo.app| & OSX\\
    \end{tabular}

    Есть две версии\ --- 32- и 64-битные, между собой не совместимы, требуют установки соответствующих \verb|.image| файлов (далее)

    \item{\verb|.image|}\\ бинарный образ запущенной \st-системы (образ памяти виртуальной машины). Файл хранит полный снимок состояния
    всех объектов в системе, формат не зависит от \term{хост-платформы}, поэтому вы можете работать с вашими данными в \win\ и \linux\ перетаскивая
    image-файлы между компьютерами (на флешке или через сетевую синхронизацию).

    По поводу \term{файлов образов} вместо исходного кода в текстовых файлах периодически возникают споры,
    что их использование было одной из причин того что \st\ умер в конце 90х на фоне взлета \cpp\ и \java.
    Если рассматривать \st\ как \emph{объектную операционную систему}, \textit{а не язык программирования}, все встает на свои места. А для резервного
    копирования ваших проектов, и обмена исходным кодом между разработчиками предусмотрены другие средства, в том числе и использование \git\ через
    файловую систему.

    С точки зрения пользователя использование образа системы очень удобно: если вы завершаете работу ВМ с сохранением, и потом перезапускаете ее
    на следующий день, состояние системы восстанавливается полностью, с точностью до пикселя экрана, положения курсора, запущенных процессов,
    и значений всех переменных\note{кроме объектов, состояние которых связано с системным временем, или перезапускается при старте виртуальной машины}.

    \item{\verb|.sources|}\\ файл исходников содержит \emph{полный исходный код \ph\ и библиотек}, включенных в релиз системы.
    Наличие файла исходников важно, так как .image-файл включает только объекты и откомпилированный байт-код их методов, но не хранит
    исходный код этих методов. Обычно .sources-файл создается один раз при релизе следующей версии системы \ph, например
    \verb|Pharo7.0-ccd1f64.sources|.

    \item{\verb|.changes|}\\ лог-файл хранит все модификации исходного кода начиная с момента создания .sources-файла, в том числе все
    изменения кода, которые делаете вы. При первоначальной установке релиза системы этот файл почти пуст, например
    \verb|Pharo7.0.4-0-64bit-ccd1f64.changes|. В результате мы имеем полную историю всех модификаций методов, можем получить диффы
    (разницу между версиями), и сделать откат кода на любой момент. Даже если вы не сохранили ваши правки\note{система была
    остановлена аварийно, или вы просто забыли сделать снапшот в .image}, при следующем запуске вы сможете восстановиться из этого файла.\\
    \verb|.sources|/\verb|.changes|\ всегда работают в паре.

\end{description}

Релиз \st-системы обязательно включает файлы .image и .changes, которые являются начальной точкой, с которой \emph{вы начинаете изменять
систему под ваши задачи}. В процессе работы они изменяются, поэтому в них должна быть обязательно разрешена запись. Так как они очень тесно
связаны между собой\note{и имеют одинаковые имена}, а .image хранит снимок состояния всех данных в системе,
эти файлы нужно исключить из под контроля антивирусной системы. Не открывайте их в редакторах, и не позволяйте их изменять внешним
программам и процессам, кроме самой виртуальной машины\ --- \emph{любое внешнее изменение этих файлов} приведет к рассогласованию и потере данных,
в том числе полностью \emph{развалит систему}.

Так как ваша работа настолько чувствительна к потере или порче этих файлов, обязательно делайте их регулярное резервное копирование
\emph{с выключенной виртуальной машиной}. Кроме того, для резервного копирования вашего кода, отслеживания изменений, и обмена с другими
программистами необходимо использовать \term{систему контроля версий}.


\clearpage
\subsecly{Настройка среды}

\fig{img/firstrun.png}{width=\textwidth}

\fig{img/settings.png}{height=.4\textheight}

\begin{itemize}
\item на скриншоте включена светлая тема для подготовки скриншотов
\item для работы рекомендуется тема \texttt{Pharo Dark}
\item шрифт увеличен чтобы этот мануал можно было читать на телефоне
\end{itemize}    

\fig{img/font.png}{height=\textheight}

\clearpage
\subsecly{Подключение репозитория \met}

Ограниченная версия \met\ описанная в этом руководстве, поставляется в составе
файлов, доступных по адресу\\
\url{https://github.com/ponyatov/Smalltalk/tree/master/metaL}\\
или локально в каталоге \file{$\sim$/Smalltalk/metaL}.

\fig{img/metacello.png}{height=.3\textheight}

\noindent
создать пустой пакет \met, \menu{World>Tools>Monticello Browser}, выбрать в фильтре \met, добавить репозиторий \menu{+Repository>filetree://},
выбрать каталог \file{$\sim$/Smalltalk/metaL}. 
В правой части появится \file{.../metaL.package}. Сделать на нем \menu{\rms>Open package>metaL.package>Load}

\fig{img/opengit.png}{height=.45\textheight}

\fig{img/metopen.png}{height=.5\textheight}

При работе с Monticello система запросит ваше имя, которое будет вставляться при создании новых
версий в системе контроля версий:

\fig{img/Author.png}{width=.7\textwidth}

\clearpage
После установки пакета через Monticello или сброса в начальное состояние
после ручного ввода кода из этого руководства требуется переинициализация
через метод \verb|MetaL.install|.

\medskip
\fig{img/plinstall.png}{height=.5\textheight}

Этод \term{метод класса}\ вызовет соответствующие методы у всех классов,
наследованных от \class{MFrame}.

\fig{img/install.png}{width=.7\textwidth}


