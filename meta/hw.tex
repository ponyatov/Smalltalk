\clearpage
\secrel{\class{HW} аппаратные компоненты}\label{hw}\secdown

\lst{meta/HW.st}{\file{HW.st}}

\fig{dot/HW.png}{width=\textwidth}

\noindent
Поддерево классов \class{HW}\ предназначено в основном для описания аппаратных систем на микроконтроллерах:
включает минимальный набор компонентов, необходимых для любого \term{оконечного узла} \iot. Задача системы
\met\ --- взять описание заданного аппаратного модуля (\term{\class{HW}-модель}), и сгенерировать
пакет исходного кода на \ci, выполняющего\note{с учетом особенностей каждой конкретной железки и 
способа ее применения}:
\begin{itemize}
    \item инициализацию аппаратных компонентов
    \item обработку прерываний
    \item интерфейс аппаратных компонентов в акторной модели \ref{akka}
    \item увязать программную логику (модель прошивки) с "железом"
    \item реализовать слой ОС жесткого реального времени (\term{RTOS}) для акторов
          (планировщик, очереди сообщений с приоритетами, сервисы управления ресурсами) \ref{rtos}
\end{itemize}

\secup
