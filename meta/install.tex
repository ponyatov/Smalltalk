\clearpage
\subsecly{Подключение репозитория \met}

Ограниченная версия \met\ описанная в этом руководстве, поставляется в составе
файлов, доступных по адресу\\
\url{https://github.com/ponyatov/Smalltalk/tree/master/metaL}\\
или локально в каталоге \file{$\sim$/Smalltalk/metaL}.

\fig{img/metacello.png}{height=.3\textheight}

\noindent
создать пустой пакет \met, \menu{World>Tools>Monticello Browser}, выбрать в фильтре \met, добавить репозиторий \menu{+Repository>filetree://},
выбрать каталог \file{$\sim$/Smalltalk/metaL}. 
В правой части появится \file{.../metaL.package}. Сделать на нем \menu{\rms>Open package>metaL.package>Load}

\fig{img/opengit.png}{height=.45\textheight}

\fig{img/metopen.png}{height=.5\textheight}

При работе с Monticello система запросит ваше имя, которое будет вставляться при создании новых
версий в системе контроля версий:

\fig{img/Author.png}{width=.7\textwidth}

\clearpage
После установки пакета через Monticello или сброса в начальное состояние
после ручного ввода кода из этого руководства требуется переинициализация
через метод \verb|MetaL.install|.

\medskip
\fig{img/plinstall.png}{height=.5\textheight}

Этод \term{метод класса}\ вызовет соответствующие методы у всех классов,
наследованных от \class{MFrame}.

\fig{img/install.png}{width=.7\textwidth}

