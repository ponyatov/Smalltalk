\clearpage
\secrel{\class{MOS} спецификация операционных систем}\label{os}\secdown

Для генерации кода очень важной характеристикой является то, под какой \term{операционной системой}
будет запускаться наш сгенерированный код. Если посмотреть на современное состояние ИТ, 
тяжела и неказиста жизнь embedded программиста: 
\begin{description}%[nosep]
    \item{\emph{большая тройка} десктопа}: \win, \linux, \macos
    \item{\emph{серверные} (склонны к плесневению)}: Solaris, AIX, ВыньServer
    \item{\emph{embedded/RTOS}}: mbed, FreeRTOS, QNX,.., bare metal\note{голое железо без RTOS}
    \item{\emph{мобильные}}: \ios, \andr
    \item{великая и ужасная} \texttt{Жаба}
\end{description}

\clearpage
Если умножить количество ОС на количество поддерживаемых платформ \ref{HW}, получим \term{матрицу целевых
систем}\ минимум на полсотни вариантов. Если вы поддерживаете несколько сборок, и пару железок поставляемых вашей компанием, не проблема. Но если
нацеливаться на полноразмерный \term{гетерогенный}\ \iot\ со всей его кашей архитектур и непредсказуемого разнообразия железа\ldots

\lst{meta/OS.st}{\file{MOS.st}}

\secrel{\class{MLinux}}\label{mlinux}\secdown

Популярная ОС общего назначения + ОС №1 для встраиваемых систем, включая мобильные телефоны на \andr,
SOHO роутеры, железо для хобби-автоматики (\rpi), ТВ-приставки, панели управления технологического оборудования, и т.д.

С некоторой адаптацией используется в стойках ЧПУ не только хоббийного класса, но и в панелях оператора профессионального
оборудования, типа Sinumeric 840D. \href{http://linuxcnc.org/}{LinuxCNC} обеспечивает полнофункциональную систему управления
для ЧПУ станков и роботов.

\bigskip
Для совместимости с \win\ --- ориентируемся на пакет \href{http://www.mingw.org/}{\mingw},
получаем возможность компиляции как в \win, так и кросс-компиляции с рабочей станции на \linux.

\clearpage
\secrel{\class{BR}: \emlin\ на основе \br}\label{br}

Хороший удобный дистрибутиво-конструктор, позволяет собрать собственный мини\linux\ для разнообразных
архитектур (x86/ARM/MIPS), включает множество пакетов, которые включаются в \emph{монолитную сборку \linux}.
Из минусов\ --- пакеты и внешние репозитории как у больших дистрибутивов и OpenWrt не поддерживаются.
Хорошо подходит для изготовления прошивок для разнообразных многофункциональных и специализированных устройств,
плохо\ --- в качестве ОС для всего что напоминает ПК общего назначения\note{в т.ч. на процессорах ARM\ --- нетбуки, \rpi,\ldots}.

\begin{itemize}
    \item \url{https://buildroot.org/}
    \item \href{https://habr.com/ru/post/448638/}{1. Общие сведения, сборка минимальной системы, настройка через меню}
    \item \href{https://habr.com/ru/post/449348/}{2. Создание конфигурации своей платы; применение external tree, rootfs-overlay, post-build скриптов}
\end{itemize}

\lst{meta/Linux.st}{\file{Linux.st}}

\secup

\secrel{\class{MRTOS} модель ОС реального времени}\label{rtos}\secdown

Для приложений \iot\ необходима система многозадачности с поддержкой \term{жесткого реального времени}.
Для этого может быть использована одна из распространенных \term{ОСРВ} \ref{mbed},\ref{freertos}, или адаптивная
модель описанная средствами \met. Поддержка mainstream OS необходима для интеграции с существующими проектами,
и для упрощения поддержки\note{чтобы вам не приходилось читать коллегам лекции по метапрограммированию вместо работы над проектами}.

Модельная ОСРВ заинтересует исследователей, в потенциале может дать адаптивность и неограниченную гибкость в конфигрировании,
но для этого и сама модель, и средства вычислений на графах в \met\ должны быть достаточно зрелыми.
Интересна реализация \term{акторного микроядра} \ref{akka} обеспечивающего RT в пределах одного узла, и прозрачную кластеризацию в
распределенной вычислительной среде.

\secrel{\class{Mbed}}\label{mbed}

\secrel{\class{MFreeRTOS}}\label{freertos}

\secup


\secup
