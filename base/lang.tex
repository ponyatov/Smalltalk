\secrel{Основные инструменты и основы языка}\secdown

\input{base/kay}

\clearpage
\secrel{Playground: командная консоль}\label{playground}

\fig{img/playground.png}{height=.5\textheight}

\medskip
\noindent\menu{World>Toold>Playgroun} или \keys{Ctrl+O+W}\\
Рабочее окно, в котором могут быть введены, отредактированы и выполнены произвольные фрагменты \st-кода.
Управление системой через код как команды, определение переменных для хранения структур данных, с которыми вы работаете.
После выделения куска кода нажать правую кнопку мыши \keys{\rms}, повится контекстное меню с доступными операциями.

\medskip\noindent

\fig{img/2p3.png}{width=.55\textwidth}

\fig{img/plmenu.png}{width=.57\textwidth}

\begin{description}
    \item{Do it} \keys{Ctrl+D} выполнить выделенный код
    \item{Print it} \keys{Ctrl+P} вычислить выражение и вывести результат под курсор
    \item{Inspect it} \keys{Ctrl+I} инспектор \ref{inspect}
\end{description}

\secrel{Inspect: просмотр объекта}\label{inspect}

\fig{img/inspect.png}{height=.5\textheight}

Просмотр и взаимодействие с конкретным объектом. Окно состоит из двух частей:
\begin{itemize}
    \item просмотр полей объекта: значения переменных
    \item нижняя часть\ --- интерактивный редактор кода аналогичный \ref{playground}
\end{itemize}

\clearpage
\secrel{Примитивы языка}\label{prim}\secdown

Примитивы\ --- минимальные элементы языка, их легко отличить по внешнему виду в исходном коде.

\secrel{Числа}

\begin{description}
    \item{целые} \class{SmallInteger} \verb|1234|
    \item{шестнадцатеричные} \class{SmallInteger} \verb|16rDeadBeef|
    \item{двоичные} \class{SmallInteger} \verb|2r1101|
    \item{c плавающей точкой} \class{SmallFloat64} \verb|12.34| \verb|1.2e3|
\end{description}

\secrel{Символы}

\class{ByteSymbol} \verb|#Symbol|

\secrel{Строки}

\secup

\input{base/vars}
\secrel{Классы и объекты}\label{oop}\secdown

\clearpage
\secrel{System Browser: редактор кода}

\fig{img/browser.png}{width=\textwidth}
\clearpage

\noindent
Наиболее часто используемое окно\ --- просмотр и редактирование классов.\\
По верхнему ряду:
\begin{itemize}[nosep]
    \item пакеты
    \item классы
    \item интерфейсы/категории
    \item методы
    \item нижняя половина: редактор кода
\end{itemize}

\fig{img/br1.png}{width=.55\textwidth}

\noindent
Начнем проект с создания пакета \met\ --- \textit{\emph{meta}programming \emph{Lang}uage}.\\
Откройте браузер \menu{World>Tools>System Browser} или \keys{Ctrl+O+B}, затем \keys{\rms}\ в области пакетов,
\menu{New package>\met>OK}. Затем создадим два новых класса: \keys{\lms}\ на новом пакете, внизу в редакторе кода
измените определение, и \term{подтвердите изменение кода} \menu{\keys{\rms}>Accept} или \keys{Ctrl+S}:

\lst{meta/MetaL.st}{MetaL.st}

Класс \class{MetaL} будет содержать служебные методы и \term{переменные класса}, 
общие для всей системы \met: процедуру инициализации и т.п.

\clearpage
\lst{meta/MFrame.st}{MFrame.st}
\lst{meta/initialize.st}{конструктор \class{MFrame}}

\fig{img/MFrame.png}{width=\textwidth}

Браузер работает в двух режимах, переключаемых под полем категорий: Instance side и Class side. 
При переключании показываются соответственно список методов экземпляра, или самого класса.

\fig{img/install.png}{width=\textwidth}

\lst{meta/install.st}{инициализация \met}


\secup

\secrel{Контейнеры данных}\label{cont}\secdown
\secup

\secrel{Циклы}\label{loops}\secdown
\secup


\secrel{Monticello: управление репозиториями}

\secup
