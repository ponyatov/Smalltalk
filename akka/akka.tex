\secrel{Акторы в распределенных системах}\label{akka}\secdown

\begin{itemize}[nosep]
\item \url{https://www.youtube.com/watch?v=7erJ1DV_Tlo}
\item \url{https://vimeo.com/79856301}
\item \href{https://www.youtube.com/watch?v=Gihdujm6EWU}{\cpp\ CoreHard Autumn 2018. Actors/CSP/Tasks \copyright\ Евгений Охотников}
\item \cite{agha}
\end{itemize}

\clearpage
В 1973 году Carl Hewitt, Peter Bishop и Richard Steiger опубликовали статью 
A Universal Modular Actor Formalism For Artificial Intelligent.

\term{Актор}\ --- это универсальная абстракция вычислительной сущности, которая в ответ на получаемое \emph{сообщение}
\begin{itemize}
\item Может \emph{отправить конечное число сообщений} другим акторам,
\item \emph{Создать} конечное число акторов,
\item \emph{Выбрать поведение} для приема следующего сообщения.
\end{itemize}

\begin{description}
\item{\textbf{коммуникация}}\ --- это \emph{асинхронный} обмен сообщениями, 
\item{\textbf{хранение}} означает, что акторы могут иметь \emph{состояние}, а 
\item{\textbf{обработка}} заключается в том, \textbf{поведение} актора определяется сообщением, 
методами его обработки, и посылкой сообщений другим акторам
\end{description}

Свойства, важные для приложений IoT и распределенных систем:
\begin{description}%[nosep]
    \item{асинхронная передача сообщений} без гарантии доставки и порядка\\
    удобный и естественный способ передачи информации в распределенных системах, в которых 
    возможны внезапные отключения узлов, неконтролируемые задержи в сети, и динамическое
    изменение структуры сети
    \item{надежный многопоточный код}
    \item{слабая связность компонентов}\\ обеспечивает маштабирование, композицию и тестирование
    \item{таймеры в виде отложенных и периодических сообщений}
    \item{чистое ООП} в формулировке Алана Кея\ --- независимые живые клетки, обменивающиеся сообщениями,
    сверхпозднее связывание компонентов программной системы
\end{description}

\secrel{Проблемы акторной модели}\secdown

\begin{itemize}[nosep]
\item \url{https://habr.com/ru/post/324420/}
\item \url{https://habr.com/ru/post/324978/}
\end{itemize}

\secrel{Домен плохо ложится на акторную модель}

Бонусы акторной модели доступны только если задача хорошо ложится на модель
\emph{асинхронной односторонней} передачи сообщений.

\secrel{Перегрузка при отсутствии backpressure}

Если актор не успевает обрабатывать входящий поток сообщений, возникает неконтрлируемый рост
очереди на обработку. При асинхронной односторонней передаче сообщений очень проблемно
организовать механизм \term{backpressure}\ --- нотификация или приостановление работы акторов
с отправляющей стороны.

Другая сложность работы в режиме перегрузки\ --- нужно затачивать защиту от перегрузке под
конкретную задачу: выбравать старые либо новые сообщения, изменять обработку, перенаправлять
поток в хранилище пока обработчик не освободится.

Как одно из решений можно применять поллинг со стороны приемника, извещающий сендеры
о готовности принимать данные, или степень загрузки очереди, чтобы сендеры могли
приостановить свою работу.

\secrel{Доставка сообщений ненадежна}

Проблема естественна, никто не гарантирует доставку и сохранение порядка сообщений.
\begin{itemize}[nosep]
\item Перепосылка сообщения после таймаута
\item Откат операции
\end{itemize}

\secrel{Слишком часто нужна синхронность}

Акторная модель чисто асинхронная, практические задачи очень часто требуют двусторонний обмен
sender/receiver, привычка программировать на синхронных вызовах делает проблему критической.

\secrel{Обеспечение прозрачной распределенности по TCP/IP}

Усложнение протоколов синхронизации, (де)сериализации, и использования распространенных
протоколов сторонних по отношению к акторной системе.

\secup


\secup
